\documentclass{article}

\usepackage{amsmath}

% local definitions
\def\exp#1{\left<#1\right>}
\title{Math Equations Used}
\author{Patrick Daley}

\begin{document}
\maketitle

\subsubsection*{Anderson-Hubbard Hamiltonian:}
\begin{align}
	H = \sum_{i,\sigma} (\epsilon_i - \mu)\hat{n}_{i \sigma} + t\sum_{\exp{i,j},\sigma} \hat{c}_{i\sigma}^{\dagger}\hat{c}_{j\sigma} + U\sum_{i}\hat{n}_{i \uparrow}\hat{n}_{i \downarrow}
\end{align}
The first and third term are the potential energy. The middle term is the kinetic energy. This is in second quantized notation.
\subsubsection*{Local Density of States}
\begin{align}
\rho_i = \sum_{\sigma,q}\big| \exp{\Psi_{q}|c_{i\sigma}^{\dagger}|\Psi_0}\big|^2 \ \delta \big(\omega - (\Omega_q - \Omega_0)\big) + \big|\exp{\Psi_{q}|c_{i\sigma}|\Psi_0}\big|^2 \ \delta \big(\omega + (\Omega_q - \Omega_0)\big) 
\end{align}
The $\delta$ are delta functions. $\Omega$ is the grand potential. $\Psi_q$ are the many-body eigenstates.
\subsubsection*{Density of states (DOS) from local density of states (LDOS):}
\begin{align}
\rho = \frac{1}{N} \sum_{i=1}^{N} \rho_i
\end{align}
\subsubsection*{Generalized inverse participation ratio (GIPR) from local density of states (LDOS):}
\begin{align}
I(\omega)= \frac{ \sum_{i=1}^{N} \rho_i^2(\omega)}{\big[ \sum_{i=1}^{N} \rho_i(\omega) \big]^2}
\end{align}
\subsubsection*{Ensemble Averaged DOS}
\begin{align}
\exp{\rho} = \frac{\sum_{i=1}^{N} \rho_j}{\Delta w \sum_{i=1}^{M} \rho_j}
\end{align}
$N$ is the number of systems in the ensemble and $M$ is the number of bins in the energy binning process. All the density of states are added together after being put into energy bins. The total area under the band is normalized to 1 by dividing it by the sum of all the bins and the width of the bins ($\Delta w$)
\subsubsection*{Ensemble Averaged GIPR}
\begin{align}
\exp{I(\omega)} = \frac{\sum_s \sum_t I_{st}(\omega)\rho_{st}(\omega)\delta(\omega-\omega_{st})}{\sum_s \sum_t \rho_{st}(\omega)\delta(\omega-\omega_{st})}
\end{align}
\end{document}