\documentclass{article}

\usepackage{graphicx,multicol}

\author{Patrick Daley}
\title{Validation of 2-Site Program}

\addtolength{\oddsidemargin}{-.875in}
\addtolength{\evensidemargin}{-.875in}
\addtolength{\textwidth}{1.75in}
\addtolength{\topmargin}{-.875in}
\addtolength{\textheight}{1.75in}

\begin{document}

\maketitle

This document outlines the validation process for the 2-site program. Each section describes a test we have run on the code to ensure the results can be trusted. They are mostly cases with certain parameters set to zero so that we can compare the results of the program to the known results at that limit. 

\section{Atomic Limit with no Interactions}
The program was run with the hopping ($t\ne0$) and the on-site interactions (U) set to zero and performed at half filling. In this case we expect the DOS to be a rectangle with edges at -W/2 and W/2 (W is the disorder strength). We expect this since when there are no onsite interactions the energies of the contributions should be the eigenvalues of the single electron hamiltonian. There are no off diagonal terms (no hopping) so the contribution energies should just be the site potentials and they are chosen from a uniform distribution bounded between -W/2 and W/2. The GIPR should be 1 between -W/2 and W/2 and then not defined anywhere else since there are no states at those energies (DOS is zero).

The graphs outputted from the program look precisely as we expected.  The DOS for one case is shown in figure 1. The GIPR had NAN (not a number) for $ | frequency | > W/2 $ because of divisions by zero but the values within the range were all 1 as expected. This data wasn't graphed because it contained no interesting information but can be viewed in the file "\textit{2-dos+ipr\_t0U0W4mu0.0.dat}" in the \textit{data} folder.

\begin{center}
	\includegraphics[width=250px]{dos2_t0u0w4.jpg} \\
	Figure 1: The DOS for the non interacting case with no hopping. Plotted from the data file \textit{2-dos+ipr\_t0U0W4mu0.0.dat}.
\end{center}

\section{Atomic Limit with Interactions}
The program was run with the hopping (t) set to zero but with the on-site interactions (U) not set to zero. We got the results we expected. They are shown in figure 2.

\begin{center}
	\includegraphics[width=250px]{dos2_t0u4w12.jpg} \\
	Figure 2: The DOS for the interacting case with no hopping. Plotted from the file \textit{2-dos+ipr\_t0U4W12mu2.0.dat}.
\end{center}

\section{Non Interacting Case}
The program was run with hopping ($t=0$) but with the on-site interactions ($U\ne0$) set to zero. In this case can compare the data to another code that solves the problem when there are no interactions (noint2.f90) as well as published work by Johri and Bhatt [1]. The non-interacting code (noint2.f90) was run with same parameters as main.f90 and the random number generator used to assign site potentials were both seeded with the same constant so the two graphs should be identical. This is the case and is shown in figure \ref{noint}.
\begin{center} 
	\includegraphics[width=250px]{dos_compareu0.jpg} \\ \label{noint}
	Figure \ref{noint}: The DOS for the non interacting case with hopping. Plotted from the data file \textit{2-dos+ipr\_t-1U0W6mu0.0.dat} and \textit{nonint\_data.dat}.
\end{center}
The two DOS graphs are identical. The the non-interacting code does output the GIPR so they cannot be compared, however both the graphs can be compared to those published in Johri and Bhatt's paper. 
\section{Comparison with Published Results}
The program is run with interactions and hopping and compared to previously published results by J. Perera and R. Wortis [2]. The simulations were not run with the same seed for the random number generator so they should not be identical but they should be very similar. The comparison of the GIPR and the DOS for one case is shown in figure \ref{int-comp}.
\begin{multicols}{2} \label{int-comp}
\begin{center}
	\includegraphics[width=250px]{dos_compareu4.jpg} \\
\end{center}
\begin{center}
	\includegraphics[width=250px]{gipr_compareu4.jpg} \\
\end{center}
\end{multicols}
\begin{center}
Figure \ref{int-comp}: The DOS and GIPR for the interacting case with hopping compared with results published by J. Perera and R. Wortis. The data for the main.f90 plot is in the data file \textit{2-dos+ipr\_t-1U4W12mu2.0.dat}.
\end{center}
The two graphs are very similar as expected and contain all the same features. The small differences can be accounted for by the noise since the two random seeds are different.
\section{Symmetry away from Half Filling}
The program is run with hopping ($t\ne0$) and onsite interactions($U\ne0$) away from half filling ($mu\ne\frac{U}{2}$). There is little published work at the moment for this simulation and it is impossible to calculate analytically however the DOS and GIPR for a filling of $0.5-x$ should be the reflection about the y axis of the DOS and GIPR with filling $0.5+x$. Figure \ref{filling} shows the DOS and GIPR for filling 0.4 and 0.6. We expect them to be reflections of each other.
\begin{multicols}{2}
\begin{center} \label{filling}
	\includegraphics[width=250px]{dos_comparef.jpg} \\ 
\end{center}
\begin{center}
	\includegraphics[width=250px]{gipr_comparef.jpg} \\
\end{center}
\end{multicols}
\begin{center}
Figure \ref{filling}: The DOS and GIPR away from half filling. Graphed using data files {2-dos+ipr\_t-1U4W12mu3.2.dat} and \textit{2-dos+ipr\_t-1U4W12mu0.8.dat}.
\end{center}
The graphs look symmetric as expected. Any small differences can be accounted for by random noise since they did not use the same seed for the random number generator
\begin{center}
	\line(1,0){400}
\end{center}
[1] S. Johri and R. Bhatt, Physical Review Letters \textbf{109}, 076402 (2012). \newline
[2] J. Perera and R. Wortis
\end{document}
